%!TEX TS-program = xelatex
\documentclass[]{friggeri-cv}
\usepackage{afterpage}
\usepackage{hyperref}
\usepackage{color}
\usepackage{xcolor}
\usepackage{smartdiagram}
\usepackage{fontspec}
% if you want to add fontawesome package
% you need to compile the tex file with LuaLaTeX
% References:
%   http://texdoc.net/texmf-dist/doc/latex/fontawesome/fontawesome.pdf
%   https://www.ctan.org/tex-archive/fonts/fontawesome?lang=en
%\usepackage{fontawesome}
\usepackage{metalogo}
\usepackage{dtklogos}
\usepackage[utf8]{inputenc}
\usepackage{tikz}
\usetikzlibrary{mindmap,shadows}
\hypersetup{
    pdftitle={},
    pdfauthor={},
    pdfsubject={},
    pdfkeywords={},
    colorlinks=false,           % no lik border color
    allbordercolors=white       % white border color for all
}
\smartdiagramset{
    bubble center node font = \footnotesize,
    bubble node font = \footnotesize,
    % specifies the minimum size of the bubble center node
    bubble center node size = 0.5cm,
    %  specifies the minimum size of the bubbles
    bubble node size = 0.5cm,
    % specifies which is the distance among the bubble center node and the other bubbles
    distance center/other bubbles = 0.3cm,
    % sets the distance from the text to the border of the bubble center node
    distance text center bubble = 0.5cm,
    % set center bubble color
    bubble center node color = pblue,
    % define the list of colors usable in the diagram
    set color list = {lightgray, materialcyan, orange, green, materialorange, materialteal, materialamber, materialindigo, materialgreen, materiallime},
    % sets the opacity at which the bubbles are shown
    bubble fill opacity = 0.6,
    % sets the opacity at which the bubble text is shown
    bubble text opacity = 0.5,
    border color=white,
}

\addbibresource{bibliography.bib}
\usepackage[dvipsnames]{xcolor}
\definecolor{pblue}{HTML}{0395DE}
\definecolor{myorange}{HTML}{F87217}

\begin{document}
\header{Gaurav}{Aradhye}
      {Apache Committer, Computer Science Graduate Student}
      
% Fake text to add separator      
\noindent\hspace{-0.16\textwidth}\makebox[\linewidth]{\rule{19.5cm}{0.4pt}}

% In the aside, each new line forces a line break
\begin{aside}
\vspace{28 mm} \leavevmode
  \section{Email}
    \textbf{aradhye.gaurav@gmail.com}
    ~
  \section{Github ID}
    \textbf{gauravaradhye}
    ~
  \section{Tel}
    +1 984-255-3707
    ~
  \section{Address}
    2504 Avent Ferry, Apt 201
    Raleigh, NC - 27606
    ~
  % use  \hspace{} or \vspace{} to change bubble size, if needed
  \section{Skillset}
        \textbf{Python, Ruby (Proficient),\\ C, Java (Intermediate),}
        Django, Ruby on Rails, \leavevmode
        Asp.net, PHP
        HTML, CSS, JS/jQuery (AngularJS)
        Android Programming
        Shell script
~
  \section{Operating Systems}
        \textbf{Linux}, OS X\leavevmode
        Microsoft Windows
~
  \section{Awards}
        \textbf{Best Innovative Application} -
        \emph{Sanitation Hackathon 2012}
        \vspace{1mm} \leavevmode
        \textbf{You Made a Difference} -
        \emph{Persistent Systems}
~

  \section{Independent Coursework}
        \textbf{Intro to Devops} -
        \emph{Udacity}
        \vspace{1mm} \leavevmode
        \textbf{Shaping up with Angular.js} -
        \emph{Code School}
~
\end{aside}
\leavevmode
\vspace{3mm}
\section{Education}
\vspace{0.8mm}
\begin{entrylist}
  \entry
    {M.S. Computer Science, 2015 - 2017 (Expected),}
    {North Carolina State University}
    {\textbf{Current GPA: 3.83}\\Courses: Operating Systems, Enterprise Storage Architecture, Design and Analysis of Algorithms, Object Oriented Design and Development, Data Driven Privacy, Building Game AI, Software Engineering, Devops\\}
  \entry
    {B.Tech. Computer Science, 2007 - 2011,}
    {Walchand College of Engineering, India}
    {Courses: Related to Computer Science, Computer Engineering and Mathematics.\\}
\end{entrylist}
\section{Experience}
\begin{entrylist}
  \entry
    {Software Engineering Intern, May 2016 - Now,}
    {Red Hat}
    {Working on ManageIQ - an open source product used to optimize and control cloud and virtualization spaces.\newline}
  \entry
    {Contributor and Committer, April 2013 - Now}
    {Apache CloudStack}
    {Managed Test Automation Framework for Apache CloudStack and downstream commercial products. Improved run time of 5000+ test cases by 15\%. Reviewed open source code and guided new test developers. Reviewed talk proposals for CloudStack-Day Conference.\\}
    \entry
    {Software Engineer, 2011 - 2013,}
    {Persistent Systems and Solutions}
    {Worked as Full Stack Developer. Actively involved in Design, Development and Management of Employee Engagement Product based on gamification.\\}
\end{entrylist}

\section{Projects}
\begin{entrylist}
\entry
    {ManageIQ }
    {Open Source}
    {Implemented Mixins to implement separation of responsibility between different software components. Built Developer Productivity Tool from scratch which gathers developer data from variety of sources and displays it in consolidated view. Technology: Ruby on Rails, Python\\}
  \entry
    {Tag based File System - TagFS }
    {URL: https://git.io/v633w}
    {TagFS is developed using FUSE (FileSystem in User Space). It allows users to tag files and directories conveniently as well as facilitates automatic tagging of files by extracting metadata. It stores tags in DAG and makes search faster. Technology: Python\\}
    \entry
    {Apache CloudStack - Test Automation Framework Development }
    {Open Source}
    {CloudStack is a Cloud Orchestration Tool adopted by Apache Software Foundation. I primarily developed new tests suits, created automation plans, monitored bugs, analyzed Jenkins reports, and improved existing code. Technology: Python\\}
    \entry
    {Analysis of highly active twitter users based on anonymity levels}
    {Research Project}
    {This research was done for mapping the behavior of twitter users against their anonymity levels. Technology: Python, Natural Language Processing Toolkit\\}
    \entry
    {SpotMe! Interactive 2D Game}
    {}
    {Designed an AI vs AI and Human vs AI Game using Processing Game Engine. Technology: Java\\}
    \entry
    {Employee Enagagement Platform - eMee}
    {}
    {Worked on front end and server side scripting. Technology used: HTML, CSS, JavaScript, Asp.net, Microsoft SQL server.\\}
    
    \entry
    {Android Weather App}
    {Independent Project}
    {Developed an android app displaying weather forecast for next 7 days.\\}
\end{entrylist}
\end{document}
